\section{Background}\label{s:background}


% \TODO{
% This section is optional. If needed, it provides the necessary context to help the reader understand the remainder of the thesis.
% }

Language interoperability; or interop for short; is broadly seen as the ability of two or more different programming languages to interact with each other and function as a part of a single system. A primary motivation behind the implementation of interop mechanics is the idea that while programming languages; especially modern ones; are often designed to be somewhat general-purpose, they are not necessarily the best tool for every job. Thus it is often the case that a system will be implemented in multiple languages. A good example of this is the tendency to separate various types of systems into frontend and backend in line with the idea of separation of concerns. In the case of web development frontend generally refers to the user interface, for which a common choice is HTML and CSS with JavaScript for interactivity. The backend on the other hand is generally responsible for the business logic and data storage, for which a common choice is Python, Java, or C\#, the interoperability in this instance takes the form of a REST API. 

The focus of this study however is on faster means of interoperability of which there are two important types in particular to focus on, Virtual Machines (VMs) and Foreign Function Interfaces (FFIs). 